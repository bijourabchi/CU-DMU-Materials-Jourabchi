\documentclass[11pt]{article}

% ---------- Packages ----------
\usepackage[margin=1in]{geometry}
\usepackage{amsmath, amssymb, amsfonts}
\usepackage{mathtools}
\usepackage{physics}        % derivatives, bras/kets, etc.
\usepackage{siunitx}        % units
\usepackage{graphicx}       % figures
\usepackage{float}          % figure placement
\usepackage{booktabs}       % nice tables
\usepackage{enumitem}       % better lists
\usepackage{fancyhdr}       % header/footer
\usepackage{titlesec}       % section formatting
\usepackage{hyperref}       % links
\usepackage{tcolorbox}      % boxes for answers
\usepackage{listings}       % code (MATLAB, Python, etc.)
\usepackage{xcolor}

% ---------- Hyperref Setup ----------
\hypersetup{
    colorlinks=true,
    linkcolor=blue,
    urlcolor=blue,
    citecolor=blue
}

% ---------- Header/Footer ----------
\pagestyle{fancy}
\fancyhf{}
\lhead{\textbf{\course}}
\rhead{\textbf{\assignment}}
\cfoot{\thepage}

% ---------- Custom Commands ----------
\newcommand{\course}{ASEN 5264 - DMU}
\newcommand{\assignment}{Homework \#2}
\newcommand{\studentname}{Bijan Jourabchi}


\newcommand{\R}{\mathbb{R}}
\newcommand{\E}{\mathbb{E}}
\newcommand{\Var}{\mathrm{Var}}
\newcommand{\Cov}{\mathrm{Cov}}

% Boxed answer environment
\newtcolorbox{answerbox}{
    colback=gray!5,
    colframe=gray!50,
    title=Answer,
    fonttitle=\bfseries
}

% Problem section format
\titleformat{\section}{\large\bfseries}{Problem \thesection:}{0.5em}{}

% ---------- Code Style ----------
\lstset{
    basicstyle=\ttfamily\small,
    keywordstyle=\color{blue},
    commentstyle=\color{green!50!black},
    stringstyle=\color{orange},
    breaklines=true,
    frame=single
}

% ---------- Document ----------
\begin{document}

% ---------- Title Block ----------
\begin{center}
    {\LARGE \textbf{\course}} \\[6pt]
    {\Large \assignment} \\[10pt]
    \studentname \\
\end{center}

\vspace{1em}
\hrule
\vspace{1em}

% =====================================================
\section{}

\begin{enumerate}[label=\alph*)]
    \item 
    $R$ (reward function) maps each state and action to a reward. In other words, for a given state $S$
    and action $A$, the reward function tells what reward we will get from taking that action. The state-action value
    function ($Q$) represents the expected return when starting in state $S$ and taking action $A$. In other words, $Q$
    can quantify our expected reward if we were to continuing with greedy policies.
    \item
    \[Q(S,A) = R(S,A) + \gamma \sum_{S'} T(S' \mid S,A ) V(S')\]
\end{enumerate}

\section{}

\section{}

\end{document}
